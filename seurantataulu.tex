\documentclass[a4paper]{article}
\usepackage{tabularx}
\usepackage{enumitem}
\usepackage[margin=1.5cm]{geometry}
\renewcommand{\familydefault}{\sfdefault}

\newcommand{\lohko}{
	\begin{tabularx}{\textwidth}{
		|c
		|X
		|>{\centering}p{1.5cm}
		|>{\centering}p{1.5cm}
		|>{\centering}p{1.5cm}
		|>{\centering}p{1.5cm}
		|>{\centering}p{1.5cm}
		|>{\centering}p{1.5cm}
		|c
		|
	}
		\hline
		& \bf Joukkue & \bf V/P & \bf V/P & \bf V/P & \bf V/P & \bf yht. & \bf kesk. & \bf sij. \\
		\hline
		1 & & / & / & / & / & / & / & \\
		\hline
		2 & & / & / & / & / & / & / & \\
		\hline
		3 & & / & / & / & / & / & / & \\
		\hline
		4 & & / & / & / & / & / & / & \\
		\hline
		5 & & / & / & / & / & / & / & \\
		\hline
	\end{tabularx}
}

\newcommand{\kaavio}[4]{
	\begin{tabularx}{\textwidth}{l X X X}
		#1 & & & \\
		\cline{2-2}
		& \multicolumn{1}{c|}{} & & \\
		\cline{3-3}
		#2 & \multicolumn{1}{c|}{} & \multicolumn{1}{c|}{} & \\
		\cline{2-2}
		 & & \multicolumn{1}{c|}{} & \\
		\cline{4-4}
		#3 & & \multicolumn{1}{c|}{} & \\
		\cline{2-2}
		 & \multicolumn{1}{c|}{} & \multicolumn{1}{c|}{} & \\
		\cline{3-3}
		#4 & \multicolumn{1}{c|}{}  & & \\
		\cline{2-2}
	\end{tabularx}
}


\begin{document}
\pagestyle{empty}
\subsection*{ Lohko 1 }
\lohko
\subsection*{ Lohko 2 }
\lohko
\subsection*{ Lohko 3 }
\lohko
\subsection*{ Lohko 4 }
\lohko
\\
\\
\noindent
\small
Jos kaksi tai useampi joukkuetta saa saman voittopistemäärän, määräytyy voittaja ja muut sijoitukset seuraavasti:
\begin{enumerate}[noitemsep]
	\item keskinäinen ottelu
	\item keskinäisten ottelujen ({\bf kesk.}) voittopistemäärä
	\item keskinäisten ottelujen ({\bf kesk.}) plus- ja miinuspisteet. Eniten plussia saanut sijoittuu paremmin.
	\item kaikki lohkon sarjapelin pelipisteet ja jos nämäkin on tasan, ratkaistaan jatkoon menijä lyhyellä uusinnalla.
\end{enumerate}
\normalsize

\subsection*{ Ylempi jatkosarja }
\small
Neljässä lohkossa pelattaessa jatkoon pääsee jokaisen lohkon voittaja. Kahdessa lohkossa pelattaessa jatkoon pääsee kummastakin lohkosta kaksi parhaiten sijoittunutta.
\normalsize
\vspace{5mm}

\noindent
\kaavio{L1-1 / L1-1}
       {L2-1 / L2-2}
       {L3-1 / L1-2}
       {L4-1 / L2-1}

\subsection*{ Alempi jatkosarja }
\kaavio{L1-2 / L1-3}
       {L2-2 / L2-4}
       {L3-2 / L1-4}
       {L4-2 / L2-3}


\end{document}
