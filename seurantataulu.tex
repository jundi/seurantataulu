\documentclass[a4paper]{article}
\usepackage[utf8]{inputenc}
\usepackage{tabularx}
\usepackage{enumitem}
\usepackage[margin=1.2cm]{geometry}
\usepackage{setspace}
\renewcommand{\familydefault}{\sfdefault}

\newcommand{\lohko}{
	\begin{tabularx}{\textwidth}{
		|c
		|X
		|>{\centering}p{1.3cm}
		|>{\centering}p{1.3cm}
		|>{\centering}p{1.3cm}
		|>{\centering}p{1.3cm}
		|>{\centering}p{1.3cm}
		|>{\centering}p{1.3cm}
		|>{\centering}p{1.3cm}
		|>{\centering}p{1.3cm}
		|c
		|
	}
		\hline
		& \bf Joukkue & \bf 1 & \bf 2 & \bf 3 & \bf 4 & \bf 5 & \bf 6 & \bf yht. & \bf kesk. & \bf sij. \\
		\hline
		1 & & --- & / & / & / & / & / & / & / & \\
		\hline
		2 & & / & --- & / & / & / & / & / & / & \\
		\hline
		3 & & / & / & --- & / & / & / & / & / & \\
		\hline
		4 & & / & / & / & --- & / & / & / & / & \\
		\hline
		5 & & / & / & / & / & --- & / & / & / & \\
		\hline
		6 & & / & / & / & / & / & --- & / & / & \\
		\hline
	\end{tabularx}
}

\newcommand{\kaavioA}[4]{
	\begin{tabularx}{\textwidth}{l X X X}
		#1 & & & \\
		\cline{2-2}
		& \multicolumn{1}{c|}{} & & \\
		\cline{3-3}
		#2 & \multicolumn{1}{c|}{} & \multicolumn{1}{c|}{} & \\
		\cline{2-2}
		 & & \multicolumn{1}{c|}{} & \\
		\cline{4-4}
		#3 & & \multicolumn{1}{c|}{} & \\
		\cline{2-2}
		 & \multicolumn{1}{c|}{} & \multicolumn{1}{c|}{} & \\
		\cline{3-3}
		#4 & \multicolumn{1}{c|}{}  & & \\
		\cline{2-2}
	\end{tabularx}
}

\newcommand{\kaavioB}[8]{
	\begin{tabularx}{\textwidth}{l X X X X}
		#1 & & & &\\
		\cline{2-2}
		& \multicolumn{1}{c|}{} \\
		\cline{3-3}
		#2 & \multicolumn{1}{c|}{} & \multicolumn{1}{c|}{} \\
		\cline{2-2}
		 & & \multicolumn{1}{c|}{} & \\
		\cline{4-4}
		#3 & & \multicolumn{1}{c|}{} & \multicolumn{1}{c|}{} \\
		\cline{2-2}
		 & \multicolumn{1}{c|}{} & \multicolumn{1}{c|}{} & \multicolumn{1}{c|}{} \\
		\cline{3-3}
		#4 & \multicolumn{1}{c|}{} & & \multicolumn{1}{c|}{} \\
		\cline{2-2}
		 & & & \multicolumn{1}{c|}{}\\
		\cline{5-5}
		#5 & & & \multicolumn{1}{c|}{}\\
		\cline{2-2}
		& \multicolumn{1}{c|}{} & & \multicolumn{1}{c|}{} \\
		\cline{3-3}
		#6 & \multicolumn{1}{c|}{} & \multicolumn{1}{c|}{} & \multicolumn{1}{c|}{} \\
		\cline{2-2}
		 & & \multicolumn{1}{c|}{} & \multicolumn{1}{c|}{} \\
		\cline{4-4}
		#7 & & \multicolumn{1}{c|}{} \\
		\cline{2-2}
		 & \multicolumn{1}{c|}{} & \multicolumn{1}{c|}{} \\
		\cline{3-3}
		#8 & \multicolumn{1}{c|}{} \\
		\cline{2-2}
	\end{tabularx}
}

\begin{document}
\pagestyle{empty}
\subsection*{Lohko 1}
\lohko
\subsection*{Lohko 2}
\lohko
\subsection*{Lohko 3}
\lohko
\subsection*{Lohko 4}
\lohko
\subsection*{Ylempi jatkosarja}
\noindent
\kaavioA{L1-1 / L1-1}
       {L2-1 / L2-2}
       {L3-1 / L1-2}
       {L4-1 / L2-1}

\subsection*{Alempi jatkosarja}
\kaavioA{L1-2 / L1-3}
       {L2-2 / L2-4}
       {L3-2 / L1-4}
       {L4-2 / L2-3}

\scriptsize
\begin{spacing}{1.0}
\vspace{2mm}
\noindent
Jos kaksi tai useampi joukkuetta saa saman voittopistemäärän, määräytyy voittaja ja muut sijoitukset seuraavasti:
\begin{enumerate}[noitemsep, nosep]
	\item keskinäinen ottelu
	\item keskinäisten ottelujen ({\bf kesk.}) voittopistemäärä
	\item keskinäisten ottelujen ({\bf kesk.}) plus- ja miinuspisteet. Eniten plussia saanut sijoittuu paremmin.
	\item kaikki lohkon sarjapelin pelipisteet ja jos nämäkin on tasan, ratkaistaan jatkoon menijä lyhyellä uusinnalla.
\end{enumerate}
\vspace{2mm}
\noindent
Neljässä lohkossa pelattaessa jatkoon pääsee jokaisen lohkon voittaja. Kahdessa lohkossa pelattaessa jatkoon pääsee kummastakin lohkosta kaksi parhaiten sijoittunutta. Kahdeksassa lohkossa pelattaessa jatkoon pääsee lohkojen voittajat.
\end{spacing}
\normalsize



\subsection*{Lohko 5}
\lohko
\subsection*{Lohko 6}
\lohko
\subsection*{Lohko 7}
\lohko
\subsection*{Lohko 8}
\lohko

\subsection*{Jatkosarja kahdeksassa lohkossa pelattaessa}
\kaavioB{L1-1}
       {L2-1}
       {L3-1}
       {L4-1}
       {L5-1}
       {L6-1}
       {L7-1}
       {L8-1}

\end{document}
